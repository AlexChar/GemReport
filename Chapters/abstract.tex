	\begin{Abstract}
		Τα πρώτα ηλεκτρονικά παιχνίδια είχαν γραφτεί εξ'ολοκλήρου σε υλισμικό. Από τότε, οι κάρτες γραφικών και οι μικροεπεξεργαστές βελτιώθηκαν, δημιουργήθηκαν κονσόλες φτιαγμένες αποκλειστικά για ηλεκτρονικά παιχνίδια, με ειδικά χειριστήρια τα οποία σου προσφέρουν διαφορετικές εμπειρίες.
		Η διαδικασία ανάπτυξης λογισμικού είναι ακριβή και ο σχεδιασμός γίνεται όλο πιο σύνθετος και περίπλοκος. Τα έργα γίνονται όλο πιο απαιτητικά και δαπανηρά. Δημιουργήθηκε η ανάγκη για ένα εργαλείο το οποίο να παρέχει ένα ομοιογενές περιβάλλον για την ανάπτυξη σύνθετων έργων. 
		Ένα CASE (Computer Aided Software Engineering) tool είναι ένα λογισμικό-εργαλείο το οποίο απλοποιεί τον κύκλο ανάπτυξης ενός λογισμικού. 
		Στο τομέα του σχεδιασμού παιχνιδιών το πιο διαδεδομένο CASE tool είναι η μηχανή γραφικών. Μια μηχανή γραφικών είναι μια σουίτα από επαναχρησιμοποιήσιμα οπτικά εργαλεία τα οποία βρίσκονται σε ένα ενιαίο περιβάλλον.
		Σκοπός της πτυχιακής είναι να αναγνωριστούν μοτίβα και τεχνικές δημιουργίας παιχνιδιών, ώστε να δημιουργηθεί ένα εργαλείο το οποίο να το προσεγγίζει από υψηλό επίπεδο με αφαιρέσεις για εύκολη μοντελοποίηση και αυτοματοποίηση κατά τη δημιουργία.
	\end{Abstract}
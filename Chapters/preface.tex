	\begin{Preface}
	\paragraph{Σκοπός} 
	Σκοπός της πτυχιακής είναι να εξηγήσει θεωρητικά και πρακτικά τα κομμάτια που απαρτίζουν μια
	τυπική μηχανή γραφικών και πώς συνδέονται αρχιτεκτονικά μεταξύ τους, μαζί με παραδείγματα και οδηγίες για την επέκτασή τους. Για το κομμάτι του rendering, δεν  έγινε απευθείας με προτόκολο που επικοινωνεί με την κάρτα γραφικών, αλλά με την ανοικτού λογισμικού βιβλιοθήκη monogame η οποία είναι και cross-platform,  δηλαδή ανάλογα με το περιβαλλον ανάπτυξης χρησιμοποιεί το ανάλογο προτόκολο για επικοινωνία με την κάρτα γραφικών, για παράδειγμα για Linux OS χρησιμοποιεί OpenGL και για Android ΟpenGL ES, για να επικεντρωθεί στο rendering γενικά και όχι για συγκεκριμένη πλατφόρμα. Τα παραδείγματα αξιοποιούν object-oriented και functional paradigms και είναι γραμμένα σε C\#.

	\paragraph{Θεμελίωση} 	
	Οι μηχανές γραφικών διαφέρουν με βάση τις λεπτομέριες αρχιτεκτονικής και υλοποίησης, αλλά τα πρότυπα σχεδίασης είναι καθολικά. Πρακτικά, οι μηχανές γραφικών απαρτίζονται από τις ίδιες βασικές έννοιες. Πολλά βιβλία έχουν γραφτεί τα οποία εστιάζουν στην κάθε έννοια ξεχωριστά αλλά ελάχιστα για το πως αυτά τα στοιχεία επικοινωνούν και αλληλεπιδρούν μεταξύ τους. 
	Η εργασία εστιάζει στην αρχιτεκτονική των μηχανών, 
	για το πώς οι ομάδες είναι οργανωμένες για να δουλεύουν μεταξύ τους, 
	ποια συστήματα και μοτίβα επαναλαμβάνονται στη δημιουργία μηχανών, 
	ποιες είναι οι απαιτήσεις για το κάθε μεγάλο υποσύστημα της μηχανής, 
	ποια συστήματα είναι αγνωστικιστικά σε παιχνίδια ή σε είδη παιχνιδιών και ποια είναι συγκεκριμένα,
	και πότε σταματά η μηχανή και ξεκινά η υλοποίηση του παιχνιδιού.
	
	Eπιπρόσθετα, θα γίνει σύγκριση με άλλες δημοφιλείς μηχανές γραφικών και βιβλιοθηκών τεχνικές για configuration management, versioning και διάφορα συστήματα ανάπτυξης. 
	\end{Preface}
	
	\begin{Acknowledgement}
		Ευχαριστίες (στο μπαμπά, στη μαμά, κτλ)
	\end{Acknowledgement}
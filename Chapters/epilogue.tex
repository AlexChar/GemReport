\chapter{Επίλoγος}
Ο σχεδιασμός ενός λογισμικού το οποίο παράγει λογισμικό είναι πολύ σύνθετος και αφηρημένος. Η υλοποίηση των υπσυστημάτων των οποίων αναφέρθηκαν στηρίζεται σε γνώση εις βάθους διάφορων τεχνολογιών όπως \gls{OpenGL}, SDL κλπ καθώς και σε γνώση ιδιαιτεροτήτων πλατφορμών, καρτών γραφικών κλπ. Κάποια από τα υποσυστήματα τα οποία δεν αναφέρθηκαν είναι τα παρακάτω:
\begin{itemize}
	\item To υποσύστημα διαχείρισης ήχου.
	\item To gameplay system το οποίο περιλαμβάνει τις λειτουργίες των μηχανισμών του παιχνιδιού.
	\item To camera system το οποίο χειρίζεται τις κάμερες του παιχνιδιού. 
	\item Το rendering system το οποίο βασίζεται σε καλή κατανόηση μαθηματικών, των σταδιών της κάρτας γραφικών (rendering pipeline) και σε πολύ χαμηλού επιπέδου βελτιστοποιήσεις για τη βέλτιστη χρήση του κύκλου του επεξεργαστή, της μνήμης και της κάρτας γραφικών.	
\end{itemize}

\section{Επεκτασιμότητα}	
	Η μηχανή προσφέρει δυνατότητες επεκτασιμότητας μέσω plugins.
	Ως plugin, ορίζεται ένα σύστημα συστατικών κάποιου λογισμικού που προσθέτει ιδιαίτερες δυνατότητες σε ένα μεγαλύτερο λογισμικό. Ένα εξειδικευμένο είδος plugin είναι το addon και περιλαμβάνει επεκτάσεις ή οπτικά θέματα. Η επέκταση μέσω plug-ins προσφέρει τα παρακάτω πλεονεκτήματα
	\begin{itemize}
		\item Δυνατότητα άλλων προγραμματιστών να προγραμματίσουν επιπλέον δυνατότητες κάποιας εφαρμογής.
		\item Υποστήριξη εύκολης πρόσθεσης νέων χαρακτηριστικών.
		\item Μείωση του μεγέθους του πυρήνα μιας εφαρμογής.
		\item Διαχωρισμός του πηγαίου κώδικα από της εφαρμογή σε περίπτωση ασύμβατων αδειών.
	\end{itemize}
	Επέκταση του Gem IDE μπορεί να γίνει χρησιμοποιώντας το Gem IDE Core και με εξαγωγή τη βιβλιοθήκης σαν Module. Παράδειγμα plugin είναι το Gem.IDE.Modules.Spritesheets το οποίο οπτικοποιεί διαδικασία δημιουργίας και εξαγωγής 2D animation spritesheets.	
	
	\subsection{Προτάσεις επέκτασης}
	Οι κυριότερες επεκτάσεις είναι οπτικοποίηση λειτουργιών...
	
\section{Συμπεράσματα}


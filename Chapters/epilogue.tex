\chapter{Επίλoγος}
Ο σχεδιασμός ενός λογισμικού το οποίο παράγει λογισμικό είναι πολύ σύνθετος και αφηρημένος. Η υλοποίηση των υπσυστημάτων των οποίων αναφέρθηκαν στηρίζεται σε γνώση εις βάθους διάφορων τεχνολογιών όπως OpenGL, SDL κλκ καθώς κα σε γνώση ιδιαιτεροτήτων πλατφορμών, καρτών γραφικών κλπ. Κάποια από τα υποσυστήματα τα οποία δεν αναφέρθηκαν είναι τα παρακάτω:
\begin{itemize}
	\item Υποσύστημα διαχείρισης ήχου
	\item Gameplay System το οποίο περιλαμβάνει τις λειτουργίες των μηχανισμων του παιχνιδιού και βρίσκεται στην κατηγορία των game-specific υποσυστημάτων.
	\item Camera System για το χειρισμό κάμερας. 
	\item Rendering system. Βασίζεται σε καλή κατανόηση μαθηματικών, των σταδιών της κάρτας γραφικών (rendering pipeline) και σε πολύ χαμηλού επιπέδου βελτιστοποιήσεις για τη βέλτιστη χρήση του κύκλου του επεξεργαστή, της μνήμης και της κάρτας γραφικών.
	\item AI Το οποίο μπορεί περιλαμβάνει από απλά finite state machines και graph path finding algorithms όπως Α*, μέχρι machine learning. 
	
\end{itemize}
\section{Επεκτασιμότητα}	
	\subsection{Plugins}
	Ως plug-in, ορίζεται ένα σύστημα συστατικών κάποιου λογισμικού που προσθέτει ιδιαίτερες δυνατότητες σε ένα μεγαλύτερο λογισμικό. Ένα εξειδικευμένο είδος plug-in είναι το add-on και περιλαμβάνει επεκτάσεις ή οπτικά θέματα.	
	\paragraph{Γιατί;}
	\begin{itemize}
		\item Δυνατότητα άλλων προγραμματιστών να προγραμματίσουν επιπλέον δυνατότητες κάποιας εφαρμογής
		\item Υποστήριξη εύκολης πρόσθεσης νέων χαρακτηριστικών
		\item Μείωση του μεγέθους του πυρήνα μιας εφαρμογής
		\item Διαχωρισμός του πηγαίου κώδικα από της εφαρμογή σε περίπτωση ασύμβατων αδειών.
	\end{itemize}
	Επέκταση του Gem IDE μπορεί να γίνει χρησιμοποιώντας το Gem IDE Core και με εξαγωγεί τη βιβλιοθήκης σαν Module. Παράδειγμα plugin είναι το Gem.IDE.Modules.Spritesheets το οποίο οπτικοποιεί διαδικασία δημιουργίας και εξαγωγής 2D animation spritesheets.
	
\section{Συμπεράματα}


\chapter{Diagnostics}
Εργαλεία για αποσφαλμάτωση και ανάπτυξη
Η ανάπτυξη του λογισμικού πρέπει να παρακολουθείται. 
>Logging / Tracing ανάλογα με το επίπεδο σημαντικότητας.
>Κανάλια πληροφόρισης για κάθε υποσύστημα.

Όταν συμβεί ένα κρίσιμο σφάλμα στο οποίο το λογισμικό τερματίζει, πρέπει να δημιουργηθεί ένα crash report και να σταλεί στους προγραμματιστές για να  εντοπίσουν το σφάλμα.
Το crash report πρέπει να περιλαμβάνει:
>To επίπεδο στο οποίο παρουσιάστηκε το σφάλμα
>Η τοποθεσία του παίχτη
>Η κατάσταση του παίχτη
>Gameplay scripts
>Exception, stack trace
>Η κατάσταση κατανομής μνήμης
>Στιγμιότυπο οθόνης

	\section{Time Ruler}
	\section{Debug Drawing API}
	Ένα υποσύστημα της μηχανής πρέπει να παρέχει λειτουργίες αποσφαλμάτωσης με απόδοση στην οθόνη και να είναι προσβάσιμο από τα υπόλοιπα συστήματα. Ο κώδικας αυτού του συστήματος πρέπει να παραλείπεται στα release builds. Αυτό επιτυχνάνεται με χρήση conditionals.
	Οι λειτουργίες αυτές περιλαμβάνουν:
	>Βασικά σχήματα
	>Σημεία
	>Σφαίρες
	>Άξονες συντεταγμένων
	>Κουτιά οριοθέτησης
	>Formatted text
	>Δυνατότητα εναλλαγής χρωμάτων
	\section{FPS Counter}
	\section{Υποσυστήματα αποσφαλμάτωσης}
	>Μενού επιλογών στο παιχνίδι με δυνατότητα εναλλαγής επιλογών και τροποποίησης τιμών
	>Debug camera
	>Κονσόλα
	>Pausing games
	>Cheats
	>Screenshots and screen captures
	>Ingame profiler: profiling blocks με αναγνώσιμα ονόματα 
	>Hierarchical Profiling
	>Εξαγωγή δεδομένων σε μορφή η οποία είναι εύκολα αναγνώσιμη από άνθρωπο για αποσφαλμάτωση.
	>Στατιστικά μνήμης και ανίχνευση διαρροών.
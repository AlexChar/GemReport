\chapter{Διαγνωστικά και Αποσφαλμάτωση}
Η ανάπτυξη του λογισμικού πρέπει να παρακολουθείται συνεχώς. Σε λογισμικά όπως ένα παιχνίδι, η επίδοση της εφαρμογής είναι εξαιρετικά σημαντική. Για τη μετάβαση από δοκιμαστικές σε τελικές εκδόσεις, χρειάζεται συλλογή και ανάλυση διαγνωστικών επιδόσεων για διαφορετικές πλατφόρμες, επεξεργαστές και κάρτες γραφικών. Ακόμη και στις τελικές εκδόσεις όμως υπάρχει η πιθανότητα σφάλματος. \cite{richter2012clr}

\section{Logging and Tracing}
Ένα logfile είναι ένα αρχείο το οποίο περιέχει γεγονότα τα οποία συμβαίνουν κατά την εκτέλεση ενός λογισμικού. Logging ονομάζουμε τη διαδικασία καταγραφής μηνυμάτων σε ένα logfile ανάλογα με το επίπεδο σημαντικότητας. Ένα εξειδικευμένο είδος logging είναι το tracing, στο οποίο καταγράφονται πληροφορίες για την εκτέλεση του προγράμματος. Χρησιμοποιείται κυρίως κατά την αποσφαλμάτωση αλλά και για συλλογή πληροφοριών με σκοπό την βελτίωση του λογισμικού για καλύτερη εμπειρία χρήστη.

Όταν συμβεί ένα κρίσιμο σφάλμα το οποίο οδηγεί στον τερματισμό του παιχνιδιού, πρέπει να δημιουργηθεί ένα crash report και να σταλεί στους προγραμματιστές για να εντοπίσουν το σφάλμα.
Το crash report πρέπει να περιλαμβάνει:
\begin{itemize}
 \item Το στιγμιότυπο του χρόνου σε UTC.
 \item To επίπεδο στο οποίο παρουσιάστηκε το σφάλμα.
 \item Την τοποθεσία του παίχτη.
 \item Την κατάσταση του παίχτη.
 \item Τα τρέχον gameplay scripts.
 \item Exception και stack trace.
 \item Την κατάσταση κατανομής μνήμης.
 \item Στιγμιότυπο οθόνης.
\end{itemize}

\section{Time Benchmarker}
Ο time ruler προσφέρει δυναμική ανάλυση του χρόνου εκτέλεσης και των ενημερώσεων με απόδοση πραγματικού χρόνου στην οθόνη με παραμετροποιήσιμα ονόματα και χρώμα μεταξύ της έκτασης υπολογισμού του. O time ruler παίρνει στιγμιότυπα του χρόνου κατά το χρόνο εκτέλεσης και τα αποδίδει ως μέσο όρο ανά δευτερόλεπτο στην οθόνη με τη μορφή γραμμής προόδου.
Η απόδοση και τα πλαίσια υπολογισμού είναι τροποποιήσιμα και ελεγχόμενα από το σύστημα διαγνωστικών για εύκολη εναλλαγή. Στο παράδειγμα \ref{lst:diagnosticsAPI} παρουσιάζεται ένα απλό παράδειγμα χρήσης του time ruler για ανάλυση αλγορίθμου.

\lstset
{
	style=sharpc, 
	caption={Time Ruler},
	label={lst:diagnosticsAPI}
}

\begin{lstlisting}
string aiRuler = "Enemy_AI_Pathfinding";
diagnostics.AddTimeRuler(rulerId,Color.Red);
diagnostics.TimeRuler[aiRuler].Show = true;

using(diagnostics.TimeRuler[aiRuler].Benchmarker)
{
	//benchmarking logic here
}
\end{lstlisting}

\section{Debug Drawing API}
Μία οντότητα η οποία αποδίδεται στην οθόνη περιέχει περισσότερες πληροφορίες από αυτό που δείχνει. Μια οντότητα αν συμμετέχει στη προσομοίωση φυσικής έχει ταχύτητα και μάζα, αν συμμετέχει στο σύστημα συγκρούσεων έχει κουτί σύγκρουσης το οποίο δείχνει τα όρια σύγκρουσης με άλλα αντικείμενα και πολλές άλλες πληροφορίες σχετικά με την κατάσταση στο παιχνίδι.
Το υποσύστημα των διαγνωστικών παρέχει λειτουργίες αποσφαλμάτωσης με απόδοση στην οθόνη και να είναι προσβάσιμο από τα υπόλοιπα συστήματα. Ο κώδικας αυτού του συστήματος παραλείπεται στα release builds. Αυτό επιτυγχάνεται με τη χρήση των conditionals τα οποία χρησιμοποιούν οδηγίες προεπεξεργαστή.

Το υποσύστημα προσφέρει δυνατότητες απόδοσης για τα παρακάτω:
\begin{itemize}
	\item Βασικά σχήματα
	\item Σημεία
	\item Σφαίρες
	\item Άξονες συντεταγμένων
	\item Κουτιά οριοθέτησης
	\item Formatted text
	\item Δυνατότητα εναλλαγής χρωμάτων
\end{itemize}

\section{Υποσυστήματα αποσφαλμάτωσης}
Το σύστημα αποσφαλμάτωσης και διαγνωστικών περιέχει πολλά υποσυστήματα. Μερικά από αυτά είναι τα παρακάτω:

\begin{itemize}
	\item Μενού επιλογών στο παιχνίδι με δυνατότητα εναλλαγής επιλογών και τροποποίησης τιμών.
	\item Debug camera η οποία κινείται χωρίς περιορισμούς στο χώρο.
	\item Κονσόλα η οποία εκτελεί scripts και εντολές υπό τη μορφή κειμένου παρόμοια με το Unix Shell.
	\item Assertions εντολές οι οποίες αν δεν αξιολογηθούν ως αληθές κατά το χρόνο εκτέλεσης του λογισμικού, τερματίζουν το λογισμικό με κωδικό σφάλματος.
	\item FPS Counter το οποίο αποδίδει στην οθόνη τον αριθμό τον \gls{FPS}.
	\item Δυνατότητα παύσης.
	\item Cheats.
	\item Screenshots και screen captures.
	\item Ιngame profiler και profiling blocks με αναγνώσιμα ονόματα.
	\item Ιεραρχικό Profiling.
	\item Εξαγωγή δεδομένων σε μορφή η οποία είναι εύκολα αναγνώσιμη από άνθρωπο για αποσφαλμάτωση.
	\item Στατιστικά μνήμης και ανίχνευση διαρροών.
\end{itemize}
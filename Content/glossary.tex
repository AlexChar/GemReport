\newglossaryentry{API}
{
	name=API,
	description={H Διεπαφή Προγραμματισμού Εφαρμογών (αγγλ. API, από το Application Programming Interface), γνωστή και ως Διασύνδεση Προγραμματισμού Εφαρμογών (για συντομία διεπαφή ή διασύνδεση), είναι η διεπαφή των προγραμματιστικών διαδικασιών που παρέχει ένα λειτουργικό σύστημα, βιβλιοθήκη ή εφαρμογή προκειμένου να επιτρέπει να γίνονται προς αυτά αιτήσεις από άλλα προγράμματα ή/και ανταλλαγή δεδομένων.}
}

\newglossaryentry{Gem Engine}
{
	name=Gem Engine,
	description={Ο πυρήνας της βιβλιοθήκης και το βασικό μέρος της πτυχιακής.}
}

\newglossaryentry{Gem IDE}
{
	name=Gem IDE,
	description={Το περιβαλλον οπτικοποίησης και χρήσης λειτουργιών της βιβλιοθήκης.}
}

\newglossaryentry{modding}
{
	name=modding,
	description={To modding αναφέρεται στην τροποποίησης του υλικού ή του λογισμικού, για να εκτελέσει μια λειτουργία η οποία δεν είχε αρχικά σχεδιαστεί ή που προορίζονται από τον σχεδιαστή.}
}

\newglossaryentry{XML}
{
	name=XML,
	description={H XML (Extensible Markup Language) είναι μία γλώσσα σήμανσης, που περιέχει ένα σύνολο κανόνων για την ηλεκτρονική κωδικοποίηση κειμένων. Ορίζεται, κυρίως, στην προδιαγραφή XML 1.0 (XML 1.0 Specification), που δημιούργησε ο διεθνής οργανισμός προτύπων W3C (World Wide Web Consortium), αλλά και σε διάφορες άλλες σχετικές προδιαγραφές ανοιχτών προτύπων.}
}

\newglossaryentry{IP}
{
	name=IP,
	description={Μία διεύθυνση IP (Ip address - Internet Protocol address), είναι ένας μοναδικός αριθμός που χρησιμοποιείται από συσκευές για τη μεταξύ τους αναγνώριση και συνεννόηση σε ένα δίκτυο υπολογιστών που χρησιμοποιεί το Internet Protocol standard. }
}

\newglossaryentry{port}
{
	name=port,
	description={Στη δικτύωσης υπολογιστών, το port (θύρα) είναι μια παράμετρος της επικοινωνίας σε ένα λειτουργικό σύστημα.}
}

\newglossaryentry{mocks}
{
	name=mocks,
	description={Στον αντικειμενοστραφή προγραμματισμό, τα mock αντικείμενα προσομοιώνουν αντικείμενα που μιμούνται τη συμπεριφορά των πραγματικών αντικειμένων με ελεγχόμενες παραμέτρους . Ένας προγραμματιστής συνήθως δημιουργεί ένα mock αντικείμενο για να ελέγξει τη συμπεριφορά κάποιου άλλου αντικειμένου.}
}

\newglossaryentry{OpenGL}
{
	name=OpenGL,
	description={H OpenGL (Open Graphics Library) είναι μια διασύνδεση προγραμματισμού εφαρμογών ανεξαρτήτου γλώσσας ή πλατφόρμας για την απόδοση δυανισματικών 2D kai 3D γραφικών. Χρησιμοποιεί τεχνικές επιτάχυνσης υλικού μέσω της μονλάδας επεξεργασίας γραφικών (GPU).}
}

\newglossaryentry{OpenGLES}
{
	name=OpenGL ES,
	description={H OpenGL ES (Open GL for Embedded Systems) είναι υποσύνολο της OpenGL και χρησιμοποιείται σε συστήματα όπως κινητά, tablets, κονσόλες κλπ.}
}


\newglossaryentry{SDK}
{
	name=SDK,
	description={Ένα SDK (Software Development Kit) είναι ένα σύνολο εργαλείων ανάπτυξης που επιτρέπουν σε έναν προγραμματιστή να δημιουργήσει λογισμικό εφαρμογών για ένα συγκεκριμένο πακέτο λογισμικού, πλατφόρμα, παιχνιδομηχανή, λειτουργικό σύστημακ κλπ.}
}

\newglossaryentry{solid}
{
	name=SOLID,
	description={Το SOLID (single responsibility, open-closed, Liskov substitution, interface segregation and dependency inversion) είναι ένα ακρώνυμο το οποίο θέσπισε ο Michael Feathers για τις πέντε πρώτες αρχές (first five principles) του Rober.C Martin στις αρχές του 2000. Όταν ένας προγραμματιστής χρησιμοποιήσει αυτές τις αρχές, θα πιο πιθανό είναι ότι το σύστημά το οποίο θα δημιουργήσεει θα είναι εύκολο να διατηρηθεί, να επεκταθεί και με λιγότερα σφάλματα.}
}

\newglossaryentry{Common Intermediate Language}
{
	name=Common Intermediate Language,
	description={Η Common Intermediate Language βρίσκεται στα χαμηλότερα επίπεδα αναγνώσιμων από ανθρωπο γλωσσών προγραμματισμού, σχεδιάστηκε από τις προδιαγραφές της Common Language Infrastructure (CLI) και χρησιμοποιείται από το .ΝΕΤ και το Μono.}
}

\newglossaryentry{UML}
{
	name=UML,
	description={Η UML (Unified Modeling Language) πλέον είναι η πρότυπη γλώσσα μοντελοποίησης στη μηχανική λογισμικού. Χρησιμοποιείται για τη γραφική απεικόνιση, προσδιορισμό, κατασκευή και τεκμηρίωση των στοιχείων ενός συστήματος λογισμικού. Μπορεί να χρησιμοποιηθεί σε διάφορες φάσεις ανάπτυξης, από την ανάλυση απαιτήσεων ως τον έλεγχο ενός ολοκληρωμένου συστήματος.}
}

\newglossaryentry{FPS}
{
	name={fps},
	description={To FPS (frames per second), είναι η συχνότητα (ρυθμός) στην οποία μία συσκευή απεικόνισης εμφανίζει διαδοχικές εικονές, τα οποία ονομάζονται πλαίσια.}
}

\newglossaryentry{GUI}
{
	name={gui},
	description={Γραφικό περιβάλλον χρήστη ή γραφική διασύνδεση/διεπαφή χρήστη (αγγλικά: Graphical User Interface, GUI ) καλείται στην πληροφορική ένα σύνολο γραφικών στοιχείων, τα οποία εμφανίζονται στην οθόνη κάποιας ψηφιακής συσκευής (π.χ. Η/Υ) και χρησιμοποιούνται για την αλληλεπίδραση του χρήστη με τη συσκευή αυτή. Παρέχουν στον τελευταίο, μέσω γραφικών, ενδείξεις και εργαλεία προκειμένου αυτός να φέρει εις πέρας κάποιες επιθυμητές λειτουργίες. Για τον λόγο αυτό δέχονται και είσοδο από τον χρήστη και αντιδρούν ανάλογα στα συμβάντα που αυτός προκαλεί με τη βοήθεια κάποιας συσκευής εισόδου (π.χ. πληκτρολόγιο, ποντίκι).}		
}

\newglossaryentry{DPI}
{
	name={dpi},
	description={Το DPI (Dots per inch) είναι μέτρο της πυκνότητας των τελειών που μπορούν να τοποθετηθούν σε μια γραμμή στο μήκος μιας ίντζας (2.54 cm).}
}

\newglossaryentry{MDA}
{
	name={MDA},
	description={To MDA είναι μια επίσημη προσέγγιση για την κατανόηση του τι είναι ένα παιχνίδι, και επιχειρεί να γεφυρώσει το χάσμα μεταξύ του σχεδιασμού, ανάπτξυης κριτικής και τεχνικής έρευνας του παιχνιδιού. }	
}

\newglossaryentry{B-Tree}
{	
	name={B-Tree},
	description={Το B-tree είναι μια αυτοεξισορροπούμενη δενδροειδή δομή δεδομένων που διατηρεί τα δεδομένα ταξινομημένα και επιτρέπει αναζητήσεις, διαδοχική πρόσβαση, εισαγωγές και διαγραφές σε λογαριθμικό χρόνο. }
}
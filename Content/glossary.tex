\newglossaryentry{API}
{
	name=API,
	description={H Διεπαφή Προγραμματισμού Εφαρμογών (αγγλ. API, από το Application Programming Interface), γνωστή και ως Διασύνδεση Προγραμματισμού Εφαρμογών (για συντομία διεπαφή ή διασύνδεση), είναι η διεπαφή των προγραμματιστικών διαδικασιών που παρέχει ένα λειτουργικό σύστημα, βιβλιοθήκη ή εφαρμογή προκειμένου να επιτρέπει να γίνονται προς αυτά αιτήσεις από άλλα προγράμματα ή/και ανταλλαγή δεδομένων.}
}

\newglossaryentry{Gem Engine}
{
	name=Gem Engine,
	description={...}
}

\newglossaryentry{Gem IDE}
{
	name=Gem IDE,
	description={...}
}

\newglossaryentry{plugins}
{
	name=Plugins,
	description={...}
}

\newglossaryentry{UML}
{
	name=UML,
	description={...}
}

\newglossaryentry{solid}
{
	name=SOLID,
	description={...}
}

\newglossaryentry{Common Intermediate Language}
{
	name=Common Intermediate Language,
	description={...}
}

\newglossaryentry{FPS}
{
	name={fps},
	description={Frame rate, also known as frame frequency, is the frequency (rate) at which an imaging device displays consecutive images called frames. The term applies equally to film and video cameras, computer graphics, and motion capture systems. Frame rate is expressed in frames per second (FPS).}
}

\newglossaryentry{GUI}
{
	name={gui},
	description={Γραφικό περιβάλλον χρήστη ή γραφική διασύνδεση/διεπαφή χρήστη (αγγλικά: Graphical User Interface, GUI ) καλείται στην πληροφορική ένα σύνολο γραφικών στοιχείων, τα οποία εμφανίζονται στην οθόνη κάποιας ψηφιακής συσκευής (π.χ. Η/Υ) και χρησιμοποιούνται για την αλληλεπίδραση του χρήστη με τη συσκευή αυτή. Παρέχουν στον τελευταίο, μέσω γραφικών, ενδείξεις και εργαλεία προκειμένου αυτός να φέρει εις πέρας κάποιες επιθυμητές λειτουργίες. Για τον λόγο αυτό δέχονται και είσοδο από τον χρήστη και αντιδρούν ανάλογα στα συμβάντα που αυτός προκαλεί με τη βοήθεια κάποιας συσκευής εισόδου (π.χ. πληκτρολόγιο, ποντίκι).}		
}

\newglossaryentry{DPI}
{
	name={dpi},
	description={Dots per inch (DPI, or dpi)[1] is a measure of spatial printing or video dot density, in particular the number of individual dots that can be placed in a line within the span of 1 inch (2.54 cm).}
}

\newglossaryentry{MDA}
{
	name={MDA},
	description={MDA is a formal approach to understanding games ñ one
		which attempts to bridge the gap between game design and
		development, game criticism, and technical game research.
		We believe this methodology will clarify and strengthen the
		iterative processes of developers, scholars and researchers
		alike, making it easier for all parties to decompose, study
		and design a broad class of game designs and game
		artifacts. }	
}

\newglossaryentry{B-Tree}
{name={B-Tree},
	description={In computer science, a B-tree is a self-balancing tree data structure that keeps data sorted and allows searches, sequential access, insertions, and deletions in logarithmic time. }
}